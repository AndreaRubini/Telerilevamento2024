\documentclass{beamer} % classe di documenti per poter creare le presentazioni che in questo linguaggio vengono definite come frame. 
\usepackage{graphicx} % Package necessario per utilizzare immagini
\usepackage{listings} % Pacchetto che può essere utilizzato per caricare degli algoritmi direttamente dagli script.
\usetheme{Copenhagen}
\usecolortheme{crane}

\title{Variazioni dei livelli idrici nell'oasi di Manzolino}

\author{Andrea Rubini}
\date{9 Agosto 2024}

\begin{document}

\maketitle

\AtBeginSection[]
{
\begin{frame}
\frametitle{Indice}    
    \tableofcontents[currentsection]
\end{frame}
}

\section{Oasi Di Manzolino}

        \begin{frame}{Oasi Di Manzolino}
            \begin{itemize}
            
                \item
                E' un'area naturalistica riconosciuta all'interno della rete comunitaria Natura 2000 come Zona di Protezione Speciale per la tutela dell'ambiente e del patrimonio faunistico. E' compresa tra tre sub-siti compresi tra la Cassa di Espansione di Manzolino, nel Comune di Castelfranco Emilia, l'ex allevamento ittico di Tivoli (frazione del Comune di San Giovanni in Persiceto, in provincia di Bologna) e l'ex allevamento ittico di Boara Rossa (nella parte nord ovest dell'area). Dal 2005 un importante intervento di bonifica e ripristino ha creato un nuovo habitat naturale visitabile attraverso percorsi e punti di osservazione specifici.
                In questo studio ci concentreremo sulla zona di Manzolino.
             
              

            \end{itemize}
        \end{frame}

        \begin{frame}{Obiettivo del progetto}
                Lo studio si pone come obbiettivo l'analisi delle variazioni dei livelli dell'acqua dal 2017 al 2024. Tutte le fotografie analizzate si riferiscono al mese di Luglio. Scopo di questo studio è analizzare se vi è una gestione dei livelli idrici finalizzata a diversificare la struttura degli specchi d'acqua ed avere zone con diverse profondità, come dichiarato dall'ente della Bonifica Burana che gestisce questo sito. 
                
        \end{frame}

        \begin{frame}{ La raccolta dati}
           Utilizzando il satellite Sentinel-2 L2A sono state raccolte 8 immagini registrate nel mese di luglio per ognuno di questi 8 anni di studio, tutte le immagini sono state raccolte tra il 17 e il 23 del mese, alla risoluzione spaziale massima possibile di 4m2/px. sono state salvate in formato tiff. a 16 bit. L'area selezionata è di 5.54Km2. Per ognuna di queste 8 foto sono state salvate le bande 02,03,04,08 e l'immagine in True color.
           Queste bande corrispondono a R,G,B e il Nir = near infrared. 
        \end{frame}
        \begin{frame}{True Color}
\begin{figure}
    \centering
    \includegraphics[width=0.5\linewidth]{5d765140-1c93-430a-b590-ee66313f5383.png}
    \caption{Immagine in true color del 2017}
    \label{fig:enter-label}
\end{figure}
                \end{frame}\section{La raccolta dati}

        \begin{frame}{Codice}
           Una volta scaricate le immagini, sono stati creati "pacchetti" di bande colore uniti per ogni anno di studio, e in seguito sostituito al verde il Nir, cosa che ha reso più immediata la visione delle zone umide, in quanto il vicino infrarosso viene totalmente assorbito dagli specchi d'acqua. in questo modo avremo immagini in false color
          
             2017
           
           B022017=rast("B04-2017.tiff")
           
           B032017=rast("B03-2017.tiff")
           
           B042017=rast("B02-2017.tiff")
           
           B082017=rast("B08-2017.tiff")
           
           U17=c(B022017,B032017,B042017,B082017)
        
        \end{frame}
        
        \begin{frame}{Codice}
            
             par(mfrow=c(4,2))
 
im.plotRGB(U17, 1,4,2)

im.plotRGB(U18, 1,4,2)

im.plotRGB(U19, 1,4,2)

im.plotRGB(U20, 1,4,2)

im.plotRGB(U21, 1,4,2)

im.plotRGB(U22, 1,4,2)

im.plotRGB(U23, 1,4,2)

im.plotRGB(U24, 1,4,2)
        \end{frame}
      
       \begin{frame}{Codice}
                            \begin{figure}
                            |
                             \includegraphics[width=0.5\linewidth]{Rplot01.jpeg}
                          \caption{False color}
                           \label{fig:enter-label}
                             \end{figure}



        \end{frame}
   
    \begin{frame}{NDWI}
           calcolo l'NDWI (Normalized Difference Water Index) seguendo la formula: 
       NDWI=(Green-Nir)/(Green+Nir)   McFeeters (1996)
      N.B.: B2 qui è la banda del verde così come impostata su sentinel. B8 è per il NIR. 
     si sfruttano queste due bande per poter mettere in risalto l'acqua. 
    

Esempio per il 2017.

DifU17 = U17[[2]] - U17[[4]] differenza tra GREEN e NIR


SumU17 = U17[[2]] + U17[[4]] somma tra GREEN e NIR

ndwiU17 = DifU17 / SumU17


          \end{frame}
       
       

 
 \begin{frame}{Classificazioni in base all'NDWI} 
        Visualizzazione delle immagini elaborate attraverso l'indice NDWI, 
suddividendoli per gli 8 anni di riferimento
La colorazione scelta è viridis cossichè la porzione di acqua verrà evidenziata in giallo, 
il resto del territorio apparirà in una scala di blu. 

        

       \end{frame}
       \begin{frame}{Classificazione in base all'NDWI}
           \begin{figure}
               \centering
               \includegraphics[width=0.5\linewidth]{4ffec1e1-43d1-4839-840e-ae4195a58977.png}
               \caption{ esempio di classificazione per l'anno 2017,Colorazione: Viridis}
               \label{fig:enter-label}
           \end{figure}
       \end{frame}
       \begin{frame}{Calcolo classi e creazione di un Dataframe}
           suddividendo in 3 cluster i risulati ottenuti con il calcolo dell'NDWI per ogni anno preso in analisi e calcolando la percentuale di copertura per ogni cluster si è potuto creare un Database con 3 differenti percentuali: Vegetazione(veg),Terreno(soil),Acqua(water)


           Calcolo le percentuali di copertura per ogni classe per il 2017

           U17c <- im.classify(ndwiU17, num.clusters=3)
           
           U17e <- ncell(U17c)
           
           F17e <- freq(U17c)
           
           prop17 = F17e / U17e
           
           perc17 = prop17 * 100
           
           perc17
           
           i17c <- im.classify(ndwiU17, num_clusters=3)
           
       \end{frame}
       
       \begin{frame}{Classificazione in base all'NDWI}

par(mfrow=c(4,2))

plot(ndwiU17, col=viridis (100))  2017 
         
           \begin{figure}
               \centering
               \includegraphics[width=0.5\linewidth]{95dabb72-5ad3-4720-90a2-96edf7b2f055.png}
               \caption{Esempio della visualizzazione della superfice utilizzando i 3 cluster scelti: Soil,Water,Veg}
               \label{fig:enter-label}
           \end{figure}
       \end{frame}

\section{Risultati}
\begin{frame}{Grafico a Barre}
anno = c(2017, 2018, 2019, 2020, 2021, 2022, 2023, 2024)

water = c(8.982954, 8.509719, 8.274400, 8.549713, 10.115000, 6.738687, 9.243336, 6.668555)

 
Calcolo delle variazioni percentuali anno su anno

variazioni percentuali = c(NA, diff(water) / head(water, -1) * 100)
 
Creazione del grafico a barre

barplot(variazioni percentuali,

+         names.arg = anno, 

+         col = ""skyblue"", 

+         main = "Variazioni Percentuali del livello idrico per Anno", 

+         xlab = "Anno", 

+         ylab = "Variazione Percentuale (%)", 

+         ylim = c(min(variazioni_percentuali, na.rm = TRUE) - 5, 

+                  max(variazioni_percentuali, na.rm = TRUE) + 5))


\end{frame}
        \begin{frame}{Variazioni percentuali del livello idrico per l'Anno }
            \begin{figure}
                \centering
                \includegraphics[width=0.5\linewidth]{3fdfb71b-c735-4b8f-ab92-035eba24fd04.png}
                \caption{Variazione percentuale di "water" in funzione dell'anno precedente}
                \label{fig:enter-label}
            \end{figure}
        \end{frame}


\section{Conclusioni}

        \begin{frame}{Conclusioni}
        Da quanto si può osservare del grafico, sapendo che la gestione del livello idrico è antropica si può dedurre che l'aumento sostanziale delle variazioni del livello di acqua dal 2021 in poi derivi principalmente da una scarsa gestione  delle acque da parte del consorzio della bonifica burana negli ultimi 3 anni.
        \end{frame}
\begin{frame}{Conclusioni}
    Utilizzando questo studio come base si potrebbe ricercare uno storico dei campionameni di specie nelle varie vasche a differenti livelli di profondità e studiare l'impatto che questi cambiamenti nei livelli idrici hanno sulle specie che abitano l'Oasi di Manzolino.
\end{frame}
\begin{frame}{Grazie per l'attenzione}
https://github.com/AndreaRubini
    \begin{figure}
        \centering
        \includegraphics[width=0.7\linewidth]{DSC_8319-Migliorato-NR copia.jpg}
        \caption{Platalea Leucorodia all'Oasi di Manzolino}
        \label{fig:enter-label}
    \end{figure}
\end{frame}
\end{document}
